% --- LaTeX CV Template - S. Venkatraman ---

% --- Set document class and font size ---

\documentclass[letterpaper, 11pt]{article}

% --- Package imports ---

\usepackage{hyperref, enumitem, longtable, amsmath, array}

% --- Page layout settings ---

% Set page margins
\usepackage[left=0.1in, right=0.1in, bottom=.8in, top=0.8in, headsep=0in, footskip=.1in]{geometry}

% Set line spacing
\renewcommand{\baselinestretch}{1.0}

% --- Page formatting settings ---

% Set link colors
\usepackage[dvipsnames]{xcolor}
\hypersetup{colorlinks=true, linkcolor=MidnightBlue, urlcolor=MidnightBlue}

% Set font to Libertine, including math support
\usepackage{libertine}
\usepackage[libertine]{newtxmath}

% Remove page numbering
\pagenumbering{gobble}

% Define font size and color for section headings
\newcommand{\headingfont}{\Large\color{OliveGreen}\vspace{-.4cm}}

% --- CV section settings ---

% Note: each section of this table (Education, Awards, Publications etc.) is 
% stored in a two-column table. The left-hand column is narrow (1 inch) and is 
% meant to store dates. The right-hand column is wide (5.2 inches) and stores 
% the main text.  Sections in which each entry might have multiple lines 
% (e.g., Education) are stored in a 'SectionTable' environment). Sections in 
% which each entry might just have one line are stored in a 'SectionTableSingleSpace'
% environment. The only difference between the two environments is the line 
% spacing between each entry. Both environments take one argument, which is the
% title of the section. See main document for how these environments are used.

% Define settings for left-hand column in which dates are printed
\newcolumntype{R}{>{\raggedleft}p{1.5in}} % Length of the left column

% Define 'SectionTable' environment
\newenvironment{SectionTable}[1]{
	\renewcommand*{\arraystretch}{1.2}
	\setlength{\tabcolsep}{10pt} % Distance between left column and main context
	\begin{longtable}{Rp{5.8in}} & #1 \\}
{\end{longtable}\vspace{-.4cm}}

% Define 'SectionTableSingleSpace' environment
\newenvironment{SectionTableSingleSpace}[1]{
	\renewcommand*{\arraystretch}{1.2}
	\setlength{\tabcolsep}{10pt}
	\begin{longtable}{Rp{5.4in}} & #1 \\[0.6em]}
{\end{longtable}\vspace{-.4cm}}

% --- Document starts here ---

\begin{document}

% --- Name and contact information ---

\begin{SectionTable}{\Huge Dafei Qin} & 
qindafei@connect.hku.hk   \newline
\href{https://dafei-qin.github.io/}{dafei-qin.github.io}
\end{SectionTable}

% --- Section: Research interests ---

% \begin{SectionTable}{\headingfont Research interests}
% & Your favorite topic, another topic, another topic, another topic, another topic
% \end{SectionTable}

% --- Section: Education ---

\begin{SectionTable}{\headingfont Education}




2020 -- 2025 & 
\textbf{Ph.D. Candidate in Computer Science}, The University of Hong Kong \newline
Supervisor: \href{https://www.cs.hku.hk/index.php/people/academic-staff/taku}{Prof. Taku Komura}, \href{https://www.cs.hku.hk/people/academic-staff/wenping}{Prof. Wenping Wang}. \newline
HKU Presidential Scholarship \\

2016 --  2020 &\textbf{B.Sc in in Electronic Engineering},  Tsinghua University \newline
 University Scholarship,  GPA: 3.7/4.0, Top 20\%. 
% --- Un-comment the next few lines if you want to include some courses you've taken ---

%& \textbf{Selected coursework}
%\begin{itemize}[itemsep=0pt, leftmargin=*]
%\item \textit{Statistics}: Asymptotic statistics, Mathematical statistics, Functional data analysis, High-dimensional statistics, Information theory
%\item \textit{Mathematics}: Measure theory, Functional analysis, Measure-theoretic probability with martingales
%\end{itemize}

\end{SectionTable}

% --- Section: Publications ---

\begin{SectionTable}{\headingfont Publications} 

2024 & 
\textbf{Instant Facial Gaussians Translator for Relightable and Interactable Facial Rendering} \newline
\textbf{Dafei Qin}, Hongyang Lin, Qixuan Zhang, Kaichun Qiao, Longwen Zhang, Zijun Zhao, Jun Saito, Jingyi Yu, Lan Xu, Taku Komura \newline
\textit{arXiv}. \newline
Translating PBR Facial assets to optimized Gaussian Splatting in seconds, enabling 30fps@1400p rendering on mobile phones. \\

2024 & 
\textbf{Media2Face: Co-speech Facial Animation Generation With Multi-Modality Guidance} \newline
Qingcheng Zhao, Pengyu Long, Qixuan Zhang, \textbf{Dafei Qin}, Han Liang, Longwen Zhang, Yingliang Zhang, Jingyi Yu, Lan Xu \newline
\textit{ACM SIGGRAPH 2024 (Conference Track)}. \newline
Co-speech facial animation generation with image and text conditions, training on the largest co-speech audio-face 3D dataset. \\

2023 & 
\textbf{Neural Face Rigging for Animating and Retargeting Facial Meshes in the Wild} \newline
\textbf{Dafei Qin}, Jun Saito, Noam Aigerman, Thibault Groueix, Taku Komura \newline
\textit{ACM SIGGRAPH 2023 (Conference Track)}. \newline
An end-to-end deep-learning approach for automatic rigging and retargeting of 3D models of human faces in the wild. 
  \\

2023 & 
\textbf{Bodyformer: Semantics-guided 3D Body Gesture Synthesis With Transformer} \newline
Kunkun Pang*, \textbf{Dafei Qin*}, Yingruo Fan, Julian Habekost, Takaaki Shiratori, Junichi Yamagishi, Taku Komura \newline
\textit{ACM SIGGRAPH 2023 (Journal Track)}. \newline
*Contributed equally. \newline
A novel transformer-based framework for automatic 3D body gesture synthesis from speech. 

 \\


\end{SectionTable}

% --- Section: Research experience ---

% \begin{SectionTable}{\headingfont Research experience}
% Month Year -- Present &
% \textbf{Title of project or lab where research was conducted} \newline
% Mentors: Professor A (University). \newline
% Description of your work. Summary of findings available \href{https://en.wikibooks.org/wiki/LaTeX/Hyperlinks}{here}. Sed dolor lacus, imperdiet non, ornare non, commodo eu, neque. Integer pretium semper justo. \\

% Month Year -- Month Year &
% \textbf{Title of project or lab where research was conducted} \newline
% Mentors: Professor B (University). \newline
% Description of your work. Summary of findings available \href{https://en.wikibooks.org/wiki/LaTeX/Hyperlinks}{here}. Sed dolor lacus, imperdiet non, ornare non, commodo eu, neque. Integer pretium semper justo. \\
% \end{SectionTable}


% --- Section: Awards, scholarships, etc ---

% \begin{SectionTableSingleSpace}{\headingfont Honors and scholarships}
% 2020 & HKU Presidential PhD Scholarship, HKU \\
% 2019 & University Scholarship, Tsinghua

% \end{SectionTableSingleSpace}

% --- Section: Industry experience ---

\begin{SectionTable}{\headingfont Industry experience}
May 2024 -- Sep 2024 &
\textbf{Adobe Inc.} -- Research Intern \newline
Controllable Video Generation\\
	
Oct 2023 -- May 2024 &
\textbf{Deemos Technology} -- Research Intern \newline
Gaussian Splatting for dynamic face\\

% Summer 2018 &
% \textbf{2012 Laboratory, Huawei} -- Research Intern \newline
% Deep learning algorithms for speech recognition on smartphone.  \\
\end{SectionTable}
% Summer 2019 &
% \textbf{Name of company (Title of job or internship)} -- City, State \newline
% Description of your responsibilities. Integer pretium semper justo. Proin risus. Nullam id quam. Nam neque. Phasellus at purus et lib ero lacinia dictum.  \\

% Summer 2018 &
% \textbf{Name of company (Title of job or internship)} -- City, State \newline
% Description of your responsibilities. Integer pretium semper justo. Proin risus. Nullam id quam. Nam neque. Phasellus at purus et lib ero lacinia dictum.  \\
% \end{SectionTable}

% --- Section: Talks and tutorials ---

% \begin{SectionTable}{\headingfont Talks and presentations}
% May 2023 & The Eleventh International Conference on Learning Representations (ICLR2023)\\
% Dec 2022 & IASC-ARS Interim Conference 2022 \\

% \end{SectionTable}

% --- Section: Teaching experience ---

% \begin{SectionTable}{\headingfont Teaching experience}
%  & Teaching assistant, COMP3271: Computer Graphics \newline
% Teaching assistant, ELEC2441: Computer Orgnization and Microprocessors \newline
% % Teaching assistant, STAT3907: Linear models and forecasting \newline
% % Teaching assistant, STAT2601: Probability and statistics I\newline
% % Teaching assistant, STAT1600: Statistics: ideas and concepts\newline
% \end{SectionTable}

% --- Section: Mentorship and service ---

% \begin{SectionTable}{\headingfont Mentorship and service}
% Month Year -- Present &
% \textbf{Title of organization you are in (Name of your role)} \newline
% Description of your responsibilities. Integer pretium semper justo. Proin risus. Nullam id quam. Nam neque. Phasellus at purus et lib ero lacinia dictum. \\

% Month Year -- Month Year &
% \textbf{Title of organization you were in (Name of your role)} \newline
% Description of your responsibilities. Integer pretium semper justo. Proin risus. Nullam id quam. Nam neque. Phasellus at purus et lib ero lacinia dictum. \\
% \end{SectionTable}

% --- Section: Professional society memberships ---

% \begin{SectionTable}{\headingfont Professional memberships}
% Year -- Present &
% Name of professional society \newline
% \textit{Short description or conferences you attended.} \\

% Year -- Present &
% Name of professional society \newline
% \textit{Short description or conferences you attended.} \\
% \end{SectionTable}
\begin{SectionTable}{\headingfont Research Interests}
	&
	Digital Avatars, Animation Synthesis.
	\end{SectionTable}

% \begin{SectionTable}{\headingfont Technical skills}
% & \textbf{Programming languages} \newline
% I use python to conduct most of the research work and I am familiar with C++ and Matlab. I use Pytorch as the main deep learning framework, and Blender as the main 3D modeling/rendering software. I use Warp4D to handle 4D facial captures.
% \end{SectionTable}
  
% --- Section: Other interests/hobbies ---

% \begin{SectionTable}{\headingfont Other interests}
% & Some of your hobbies, etc.
% \end{SectionTable}

% --- End of CV! ---

\end{document}





